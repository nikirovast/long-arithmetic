% Diese Zeile bitte -nicht- aendern.
\documentclass[course=erap]{aspdoc}

%%%%%%%%%%%%%%%%%%%%%%%%%%%%%%%%%
%% TODO: Ersetzen Sie in den folgenden Zeilen die entsprechenden -Texte-
%% mit den richtigen Werten.
\newcommand{\theGroup}{147} % Beispiel: 42
\newcommand{\theNumber}{A326} % Beispiel: A123
\author{Yulia Nikirova \and Andriy Manucharyan}
\date{Sommersemester 2021} % Beispiel: Wintersemester 2019/20
%%%%%%%%%%%%%%%%%%%%%%%%%%%%%%%%%

% Diese Zeile bitte -nicht- aendern.
\title{Gruppe \theGroup{} -- Abgabe zu Aufgabe \theNumber}

\begin{document}
\maketitle

\section{Einleitung}
Die vorliegende Arbeit beschäftigt sich mit dem Algorithmus der schnellen Exponentiation von Matrizen von großen Zahlen und seine Anwendung auf die iterative Berechnung die Konstante  $\sqrt{2}$ mit der beliebigen vom Benutzer wählbaren Genauigkeit. Zu diesem Zweck wurde eine Methode zur Speicherung langer Zahlen und zur effizienten Durchführung mathematischer Operationen auf ihnen entwickelt und die Anzahl der Operationen, die zum Erreichen einer bestimmten Genauigkeit erforderlich sind, mathematisch abgeschätzt.   


\section{Lösungsansatz}


% TODO: Je nach Aufgabenstellung einen der Begriffe wählen
\section{Genauigkeit}


\section{Performanzanalyse}


\section{Zusammenfassung und Ausblick}

% TODO: Fuegen Sie Ihre Quellen der Datei Ausarbeitung.bib hinzu
% Referenzieren Sie diese dann mit \cite{}.
% Beispiel: CR2 ist ein Register der x86-Architektur~\cite{intel2017man}.
\bibliographystyle{plain}
\bibliography{Ausarbeitung}{}

\end{document}
